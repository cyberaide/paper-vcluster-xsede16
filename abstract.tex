\begin{abstract}
Hardware virtualization has been gaining a significant share of computing time
in the last years. Using virtual machines (VMs) for parallel computing is an
attractive option for many users. A VM gives users a freedom of choosing an
operating system, software stack and security policies, leaving the physical
hardware, OS management, and billing to physical cluster administrators. The
well-known solutions for cloud computing, both commercial (Amazon Cloud, Google
Cloud, Yahoo Cloud, etc.) and open-source (OpenStack, Eucalyptus) provide
platforms for running a single VM or a group of VMs. With all the benefits,
there are also some drawbacks, which include reduced performance when running
code inside of a VM, increased complexity of cluster management, as well as the
need to learn new tools and protocols to manage the clusters.

At SDSC, we have created a novel framework and infrastructure by providing virtual
HPC clusters to projects using the NSF sponsored {\em Comet} supercomputer.
Managing virtual clusters on {\em Comet} is similar to managing a bare-metal 
cluster in terms of processes and tools that are employed. This is beneficial because
such processes and tools are familiar to cluster administrators. Unlike
platforms like AWS, {\em Comet}'s virtualization capability supports
installing VMs from ISOs (i.e., a CD-ROM or DVD image) or via an isolated
management VLAN (PXE). At the same time, we're helping projects take advantage
of VMs by providing an enhanced client tool for interaction with our management
system called Cloudmesh client. Cloudmesh client can also be used to manage
virtual machines on OpenStack, AWS, and Azure.

The article describes our design and approach to running virtual clusters, the
tools we developed, and initial user experience.
\end{abstract}
