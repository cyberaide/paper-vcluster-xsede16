%!TEX root = virtual-clusters.tex

\subsection{Terminology}\label{S:terminology}

As the concept of using standard HPC infrastructure to manage virtual
clusters is new and introduces abstractions that may not be available
by standard virtualization frameworks such as OpenStack, AWS, and
Azure, we provide the necessary terminology that we will use
throughout this paper. The biggest difference we have is in the
introduction of computesets. The definitions include:

\begin{description}

\item[Node:] The term node is used to refer to individual computers in a virtual
cluster. The term node is synonymous with Virtual Machine (VM).

\item[Compute:] A node with substantial computational resources used to perform
work in a virtual cluster. Compute Nodes (CN) are started and stopped on request
by the cluster administrator.

\item[Frontend:] A node with limited computational resources used to manage a
virtual cluster. Frontend Nodes (FN) typically remain running 24 hours a day
and can be started and stopped on request by the cluster administrator

\item[Virtual cluster:] A virtual cluster (VC) is a loosely or tightly connected
network of FN and CNs managed together by a virtual cluster administrator.

\item[Computeset:] A group of CNs started together and being in some state
(submitted, started, finished, failed). Each CN can only belong to 1 {\em
computeset\/} in submitted or active state. Compute sets can be merged
to build a larger virtual cluster on demand.

\item[Image:] A file containing the contents and structure (ISO9660) of a disk
volume which can be attached as a cdrom to a node.

\item[Image attach:] Attach is an action applied to a node and image pair
whereby the contents of the image are made available to a node on the next power
on.

\item[Image detach:] Detach is an action applied to a node and image pair
whereby the contents of the image are made unavailable to the node on the next
power on.

\item[Console:] An interactive representation of the screen of a node (text or
graphical) provided to assist with node installation and management.

\end{description}
