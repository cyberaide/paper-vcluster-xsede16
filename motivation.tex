%!TEX root = virtual-clusters.tex

\subsection{Motivation} \label{S:motivation}

The reasons for creating the novel {\em Comet} virtual environment are
to define an efficient system architecture for virtual HPC clusters,
provide VC performance comparable to the underlying physical system,
and to deliver an HPC platfrom to the VC users and administrators in a
familiar fashion. VMs that are started on {\em Comet} are running on the same servers as regular
HPC jobs and VMs consequently have access to the same high-performance
hardware, including the InfiniBand interconnect via SR-IOV, local flash
drives, significant amounts of RAM and multicore CPUs with vector
instruction sets. Combined with the semantics and management
features we emphasize through our client interfaces we deliver near bare-metal
experience to the users and administrators of the VCs.  Hence the
abstraction of a virtual cluster mimics that of a physical one.

Two of the features that enable this bare-metal approach for
management are the private management VLAN and installation from an
ISO image. Each VC has its own VLAN, enabling the PXE-boot from a
frontend server and full control of an associated local network.  Via
this abstraction clusters can be management using existing tools
mentioned earlier, in addition to Cobbler \cite{cobbler} which also handles
bare-metal provisioning over a local network. Furthermore
we can leverage other commercial cluster management software as well,
like Bright Cluster Manager \cite{bright}.

For installation and troubleshooting, a VM can be booted up from a
regular ISO image uploaded to {\em Comet}'s
image server and mounted on the desired server on request. Using an
ISO is for many the easiest way to install a frontend of a cluster---a
machine running 24/7 and providing access and management for other
nodes. Users can also start a compute nodes using a regular ISO, or
PXE-boot from the frontend if an appropriate cluster management tool
has been set up. Hence compute nodes can be installed with minimum
effort while providing access to the desired OS supported by the KVM
hypervisor. There is no need to conduct complex manipulations of
images or limit use to a handful of sanctioned images. Users can
control the installation process by opening the {\em Comet} VM
console in a browser with the cloudmesh client tool (see Section \ref{S:cloudmesh}).

HPC admins have extensive experience working with bare-metal
clusters and the existing tools. Switching to new paradigms could require significant
resources in time, hardware and money spent to adopt such new
approaches. This also includes finding or writing new tools to adapt to
new system frameworks. {\em Comet}'s virtual environment makes running virtual
clusters straightforward by leveraging existing expertise of the
users. Using such well known tools will avoid the significant learning
curve associated for example with OpenStack as obvious from
\cite{ossurvey2016}.

The disk image management on {\em Comet} is a unique feature that
reduces the necessary storage and network infrastructure required
while also providing VMs with fast local storage. In short, the KVM
disk images for the VMs in a cluster are stored on a central server,
but migrated to the container while the VM is running and flushed back
when the VM is shut down. This lowers the effort needed for the
centralized storage server and limits the network bandwidth utilized
to transferring the image. This design also provides VMs with disk
performance based on the SSDs within the physical compute nodes.

In addition to production use of {\em Comet} VCs, the near bare-metal
approach makes {\em Comet} them an excellent platform for 
training. Groups can learn how to use tools that work on their real
systems in a controlled VM environment. With our solutions, we
demonstrate that VCs are 
easy to set up, replicate, reinstall, and destroy. A new
cluster can be booted up for a class and quickly torn down after its
intended use \cite{caida}. Hence, we provide an efficient platform for
trainees that need to get experience with software and services
supported by access which is very close to the real
hardware. Consequently, the huge benefit of our {\em Comet virtual
  clusters} is the combination of near bare-metal and VM features
while utilizing existing HPC batch systems.
