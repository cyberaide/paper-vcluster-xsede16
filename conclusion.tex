
%%% Local Variables:
%%% mode: latex
%%% TeX-master: t
%%% End:


\section{Conclusions} \label{S:conclusion}

We have proven that it is possible to integrate virtual machine
management leveraging an existing HPC batch queuing systems as part of the
operation of a major national scale compute infrastructure. This novel technology solution and
operational model provides the ability to integrate best practices
from both approaches, traditional supercomputing and utilizing
virtual machines to further the long tail of science. Overall
management of such a system is reduced as it
is well integrated in to HPC administrators familiar toolset. This
applies both to the operation of {\em Comet} and the virtual clusters
it hosts.

Additionally, we developed a powerful client interface to {\em Comet}'s
virtualization system.  The Cloudmesh client was developed with the
goal to interface easily and to enable its users to have access
programmatically to {\em Comet} via a Python API, a command shell, a
commandline tool and a prototype portal. Furthermore as new features
become available the client provides an easy mechanism to be updated.

We have utilized the tools and capabilities provided in this paper in a
variety of different scenarios and shown that they are beneficial for
the particular effort. As part of our efforts new use cases, namely
training and development, have emerged.

We will expand upon our current already operational prototypes and
integrate them into a production system that will be available
soon (during Q2 2016). Users that are interested today to explore the virtual modality of
{\em Comet} or Cloudmesh client for comet should contact the authors of the
paper.
